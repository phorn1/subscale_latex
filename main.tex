% !TeX encoding = UTF-8
% !TeX spellcheck = de_DE

%% Dies gibt Warnungen aus, sollten veraltete LaTeX-Befehle verwendet werden
\RequirePackage[l2tabu, orthodox]{nag}

\documentclass[utf8,biblatex]{lni}
\bibliography{lni-paper-example-de}

%% Schöne Tabellen mittels \toprule, \midrule, \bottomrule
\usepackage{booktabs}

%% Zu Demonstrationszwecken
\usepackage[math]{blindtext}
\usepackage{mwe}

%% BibLaTeX-Sonderkonfiguration,
%% falls man schnell eine existierende Bibliographie wiederverwenden will, aber nicht die .bib-Datei händisch anpassen möchte.
%% Bitte \iffalse und \fi entfernen, dann ist diese Konfiguration aktiviert.

\iffalse
\AtEveryBibitem{%
  \ifentrytype{article}{%
  }{%
    \clearfield{doi}%
    \clearfield{issn}%
    \clearfield{url}%
    \clearfield{urldate}%
  }%
  \ifentrytype{inproceedings}{%
  }{%
    \clearfield{doi}%
    \clearfield{issn}%
    \clearfield{url}%
    \clearfield{urldate}%
  }%
}
\fi

\begin{document}
%%% Mehrere Autoren werden durch \and voneinander getrennt.
%%% Die Fußnote enthält die Adresse sowie eine E-Mail-Adresse.
%%% Das optionale Argument (sofern angegeben) wird für die Kopfzeile verwendet.
\title[Subscale]{Zusammenfassung des Subscale-Algorithmus}
%%%\subtitle{Untertitel / Subtitle} % falls benötigt
\author[Jonas Kienzle, Pascal Kunkel \and Pius Horn]
{Jonas Kienzle\footnote{Hochschule Offenburg, EMI, Straße, Postleitzahl Ort, Land \email{emailaddress@author1}} \and
 Pascal Kunkel\footnote{Hochschule Offenburg, EMI, Address, Country \email{emailaddress@author2}} \and
Pius Horn\footnote{Hochschule Offenburg, EMI, Schubertstraße 5, 77948, Friesenheim, Deutschland \email{phorn2@stud.hs-offenburg.de}}}
%\startpage{} % Beginn der Seitenzählung für diesen Beitrag
%\editor{Herausgeber et al.}    % Namen der Herausgeber
%\booktitle{Name-der-Konferenz} % Name des Tagungsband; optional Kurztitel
%\yearofpublication{2022}
%%%\lnidoi{18.18420/provided-by-editor-02} % Falls bekannt
\maketitle

\begin{abstract}
Die \LaTeX-Klasse \texttt{lni} setzt die Layout-Vorgaben für Beiträge in LNI Konferenzbänden um.
Dieses Dokument beschreibt ihre Verwendung und ist ein Beispiel für die entsprechende Darstellung.
Der Abstract ist ein kurzer Überblick über die Arbeit der zwischen 70 und 150 Wörtern lang sein und das Wichtigste enthalten sollte.
Die Formatierung erfolgt automatisch innerhalb des abstract-Bereichs.
\end{abstract}

\begin{keywords}
Subscale \and DBSCAN
\end{keywords}

\section{Verwendung}



%% \bibliography{lni-paper-example-de} ist hier nicht erlaubt: biblatex erwartet dies bei der Preambel
%% Starten Sie "biber paper", um eine Biliographie zu erzeugen.
\printbibliography

\end{document}
