% !TeX encoding = UTF-8
% !TeX spellcheck = de_DE

%% Dies gibt Warnungen aus, sollten veraltete LaTeX-Befehle verwendet werden
\RequirePackage[l2tabu, orthodox]{nag}

\documentclass[utf8,biblatex]{lni}
\bibliography{lni-paper-example-de}

%% Schöne Tabellen mittels \toprule, \midrule, \bottomrule
\usepackage{booktabs}

%% Zu Demonstrationszwecken
\usepackage[math]{blindtext}
\usepackage{mwe}

%% BibLaTeX-Sonderkonfiguration,
%% falls man schnell eine existierende Bibliographie wiederverwenden will, aber nicht die .bib-Datei händisch anpassen möchte.
%% Bitte \iffalse und \fi entfernen, dann ist diese Konfiguration aktiviert.

\iffalse
\AtEveryBibitem{%
  \ifentrytype{article}{%
  }{%
    \clearfield{doi}%
    \clearfield{issn}%
    \clearfield{url}%
    \clearfield{urldate}%
  }%
  \ifentrytype{inproceedings}{%
  }{%
    \clearfield{doi}%
    \clearfield{issn}%
    \clearfield{url}%
    \clearfield{urldate}%
  }%
}
\fi

\begin{document}
%%% Mehrere Autoren werden durch \and voneinander getrennt.
%%% Die Fußnote enthält die Adresse sowie eine E-Mail-Adresse.
%%% Das optionale Argument (sofern angegeben) wird für die Kopfzeile verwendet.
\title[Subscale]{Zusammenfassung des Subscale-Algorithmus}
%%%\subtitle{Untertitel / Subtitle} % falls benötigt
\author[Jonas Kienzle, Pascal Kunkel \and Pius Horn]
{Jonas Kienzle\footnote{Hochschule Offenburg, EMI, Heimgasse 23a, 77797, Ohlsbach, Deutschland \email{jkienzle@stud.hs-offenburg.de}} \and
 Pascal Kunkel\footnote{Hochschule Offenburg, EMI, Im Langenacker 7, 76534, Baden-Baden, Deutschland \email{pkunkel1@stud.hs-offenburg.de}} \and
Pius Horn\footnote{Hochschule Offenburg, EMI, Schubertstraße 5, 77948, Friesenheim, Deutschland \email{phorn2@stud.hs-offenburg.de}}}
%\startpage{} % Beginn der Seitenzählung für diesen Beitrag
%\editor{Herausgeber et al.}    % Namen der Herausgeber
%\booktitle{Name-der-Konferenz} % Name des Tagungsband; optional Kurztitel
%\yearofpublication{2022}
%%%\lnidoi{18.18420/provided-by-editor-02} % Falls bekannt
\maketitle

\begin{abstract}
Die \LaTeX-Klasse \texttt{lni} setzt die Layout-Vorgaben für Beiträge in LNI Konferenzbänden um.
Dieses Dokument beschreibt ihre Verwendung und ist ein Beispiel für die entsprechende Darstellung.
Der Abstract ist ein kurzer Überblick über die Arbeit der zwischen 70 und 150 Wörtern lang sein und das Wichtigste enthalten sollte.
Die Formatierung erfolgt automatisch innerhalb des abstract-Bereichs.
\end{abstract}

\begin{keywords}
Subscale \and DBSCAN \and
\end{keywords}

\section{Einleitung}
Eine wichtige Herausforderung in Datascience ist es in großen
Datenmengen Zusammenhänge in Form von Anhäufungen ähnlicher Daten
zu erkennen. Dies wird auch Cluster genannt.
Dieser Bereich der Datascience findet unter anderem in den
Domänen Biologie, maschinelles Sehen, Astronomie und Auswertung
von personellen Daten Anwendung.
In der Medizintechnik können solche Algorithmen genutzt werden
um Risikogruppen zu identifizieren.
Dabei stellt ein Patient einen Datensatz dar und seine
Gesundheitsdaten wie das Blutbild, Alter und Geschlecht
die entsprechenden Attribute.
Jedes Attribut ist in diesem Fall eine Dimension.
Das Ziel ist es in einer großen Menge von
Patienten Zusammenhänge zwischen den Attributen zu finden.
Mit steigender Anzahl von Dimensionen wird es schwierig
zusammenhängende Cluster zu ermitteln,
da bei einer großen Menge von Attributen die einzelnen
Attribute nicht so sehr ins Gewicht fallen.
Deswegen wird versucht die unwichtigen Attribute auszublenden.
Man sucht also in einem Subspace nach Clustern.
Mit der Anzahl der Dimensionen wächst die Anzahl der
möglichen Subspaces exponentiell schnell, was die
Suche nach relevanten Subspaces sehr rechenintensiv
werden lässt. Hier setzt der Subscale-Algorithmus an,
da dieser mit hochdimensionalen Datensets besser wie
vergleichbare Algorithmen wie CLIQUE, SUBCLU und INSCY
skaliert.

\section{DBSCAN}

\subsection{Funktionsweise}
\section{Subscale Clustering}
\subsection{Funktionsweise}
\section{Fazit}



%% \bibliography{lni-paper-example-de} ist hier nicht erlaubt: biblatex erwartet dies bei der Preambel
%% Starten Sie "biber paper", um eine Biliographie zu erzeugen.
\printbibliography

\end{document}
